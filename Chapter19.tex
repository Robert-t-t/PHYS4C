\documentclass{article}
\usepackage{amsmath, amssymb}
\usepackage{pgfplots}
\usepackage{tikz}
\usepackage{booktabs}


% page 541 has a good amout of information

\begin{document}


\title{Chapter 19}
\author{}
\date{}

\maketitle



\section*{What is the frist law of Thermodynamics?}

To sum it up the frist law of Thermodynamic is energy transfering into/out of the system as work/heat



\begin{equation}
    \Delta E_{\text{th}} = W + Q
\end{equation}

where:
\begin{itemize}
    \item $\Delta E_{\text{th}}$ = Change in thermal energy (J)
    \item $W$ = Work done on the system (J)
    \item $Q$ = Heat added to the system (J)
\end{itemize}


\section*{19.2 Work in Ideal-Gas Processes}

\begin{equation}
    W = - \int_{V_i}^{V_f} P \, dV
\end{equation}

\section*{Isochoric Process}

\begin{equation}
    W = 0
\end{equation}
Isochoric process is a process in which the volume of the system is constant and the work done is zero.


\section*{Isobaric Process}

    \begin{equation}
        W = - P \Delta V
    \end{equation}
    Isobaric process is a process in which the pressure of the system is constant and the work done is $-P \Delta V$.
    where:
    
    
    \begin{itemize}
        \item $W$ = Work done on the system (J)
        \item $P$ = Pressure of the gas (Pa)
        \item $\Delta V$ = Change in volume of the gas (m$^3$)
    \end{itemize}

\section*{isothermal Process}

\begin{equation}
    W = - nRT \ln \left( \frac{V_f}{V_i} \right) = - p_i V_i \ln \left( \frac{V_f}{V_i} \right) = - p_f V_f \ln \left( \frac{V_f}{V_i} \right)
\end{equation}

where:
\begin{itemize}
    \item $n$ = Number of moles of the gas (mol)
    \item $R$ = Ideal gas constant (8.31 J/mol $\cdot$ K)
    \item $T$ = Temperature of the gas (K)
    \item $V_i$ = Initial volume of the gas (m$^3$)
    \item $V_f$ = Final volume of the gas (m$^3$)
    \item $p_i$ = Initial pressure of the gas (Pa)
    \item $p_f$ = Final pressure of the gas (Pa)
\end{itemize}


\vspace{1cm}

\begin{equation}
    \Delta E_{\text{th}} = 0
\end{equation}


\section*{Adiabatic Process}
Def: An adiabatic process is a process in which no heat is added to or removed from the system.

\section*{Heat}

to be continue

\section*{19.3.2 Units of Heat}

\begin{equation}
    1 \text{ cal} = 4.186 \text{ J}
\end{equation}


\vspace{2cm}




\begin{table}[h]
    \centering
    \section*{19.5.1 Specific Heat and molar specifc heats of solids and liquids}
    \begin{tabular}{lcc}
        \toprule
        \textbf{Substance} & \textbf{$c$ (J/kg K)} & \textbf{$C$ (J/mol K)} \\
        \midrule
        \multicolumn{3}{l}{\textbf{Solids}} \\
        Aluminum  & 900  & 24.3 \\
        Copper    & 385  & 24.4 \\
        Iron      & 449  & 25.1 \\
        Gold      & 129  & 25.4 \\
        Lead      & 128  & 26.5 \\
        Ice       & 2090 & 37.6 \\
        \midrule
        \multicolumn{3}{l}{\textbf{Liquids}} \\
        Ethyl alcohol & 2400 & 110.4 \\
        Mercury       & 140  & 28.1 \\
        Water         & 4190 & 75.4 \\
        \bottomrule
    \end{tabular}
\end{table}

\vspace{2cm}


\begin{equation}
    Q = Mc \Delta T
\end{equation}

where:
\begin{itemize}
    \item $Q$ = Heat added to the system (J)
    \item $M$ = Mass of the substance (kg)
    \item $c$ = Specific heat of the substance (J/kg K)
    \item $\Delta T$ = Change in temperature of the substance (K)
\end{itemize}

\vspace{2cm}

\begin{equation}
    Q = nC \Delta T
\end{equation}

where:
\begin{itemize}
    \item $Q$ = Heat added to the system (J)
    \item $n$ = Number of moles of the substance (mol)
    \item $C$ = Molar specific heat of the substance (J/mol K)
    \item $\Delta T$ = Change in temperature of the substance (K)
\end{itemize}


\section*{19.5.2 Phase Changes and Heat of Transformation}



\section*{Addtional Notes}
when you want to find M (molar mass) you can use the following formula:
\begin{equation}
    M = \frac{m}{n}
\end{equation}

Find the number from the periodic table and divide it by the number of moles of the gas to find the molar mass of the gas.

where:
\begin{itemize}
    \item $M$ = Molar mass (kg/mol)
    \item $m$ = Mass of the gas (kg) usally given in grams
    \item $n$ = Number of moles of the gas (mol)
\end{itemize}


\section*{Finding the number of moles of the gas}
\begin{equation}
    n = \frac{\text{mass of the gas}}{\text{molar mass of the gas}}
\end{equation}


\end{document}