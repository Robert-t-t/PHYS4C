\documentclass{article}
\usepackage{amsmath, amssymb}

\begin{document}
\section*{Equations to help with Exam 2, PHY-4C}
\subsection*{Important Information}
Monatomic gases have \( r = 0.5 \times 10^{-10} \) m and diatomic molecules have \( r = 1.0 \times 10^{-10} \) m.

\subsection*{Important}

\textbf{Conversion from grams to kilograms:}
\begin{equation}
1 \, \text{g} = 1 \times 10^{-3} \, \text{kg}
\end{equation}

\textbf{Conversion of cubic centimeters to cubic meters:}
\begin{equation}
1 \, \text{cm}^3 = 1 \times 10^{-6} \, \text{m}^3
\end{equation}

\textbf{Boltzmann constant:}
\begin{equation}
k_B = 1.38 \times 10^{-23} \, \text{J/K}
\end{equation}

\textbf{Conversion from Celsius to Kelvin:}
\begin{equation}
T(K) = T(°C) + 273.15
\end{equation}

\textbf{Conversion from Kelvin to Celsius:}
\begin{equation}
T(°C) = T(K) - 273.15
\end{equation}

\textbf{Avogadro's number:}
\begin{equation}
N_A = 6.022 \times 10^{23} \, \text{mol}^{-1}
\end{equation}

\textbf{Conversion of kilopascals to pascals:}
\begin{equation}
1 \, \text{kPa} = 1 \times 10^3 \, \text{m}^3
\end{equation}

\subsection*{Mean Free Path Equation}
\begin{equation}
\lambda = \frac{1}{4\sqrt{2} \pi {N}/{V} r^2}
\end{equation}
where:
\begin{description}
    \item[$\lambda$] Mean free path (m)
    \item[$N$] Number of molecules (molecules)
    \item[$V$] Volume (m\(^3\))
    \item[$r$] Radius of the molecules (m)
\end{description}


\textbf{Number density:}
\begin{equation}
\frac{N}{V} = n
\end{equation}
where:
\begin{description}
    \item[$N$] Number of molecules (molecules)
    \item[$V$] Volume (m\(^3\))
    \item[$n$] Number density (molecules/m\(^3\))
\end{description}

\subsection*{Root Mean Square (RMS) Velocity}
\begin{equation}
    v_{\text{rms}} = \sqrt{\frac{3k_B T}{m}}
    \end{equation}
where:
\begin{description}
    \item[$v_{\text{rms}}$] Root mean square velocity (m/s)
    \item[$k$] Boltzmann constant (J/K)
    \item[$T$] Temperature (K)
    \item[$m$] Mass of a molecule (kg)
\end{description}


\subsection*{Pressure on the Wall of the Container}
\begin{equation}
P = \frac{F_{\text{on wall}}}{A} = \frac{1}{3V} m v_{\text{rms}}^2
\end{equation}
where:
\begin{description}
    \item[$P$] Pressure (Pa)
    \item[$F_{\text{on wall}}$] Force on the wall (N)
    \item[$A$] Area of the wall (m\(^2\))
    \item[$V$] Volume of the container (m\(^3\))
    \item[$m$] Mass of a molecule (kg)
    \item[$v_{\text{rms}}$] Root mean square velocity (m/s)
\end{description}

\subsection*{Translational Kinetic Energy}
\begin{equation}
\epsilon_{\text{trans}} = \frac{1}{2} m v_{\text{rms}}^2
\end{equation}
where:
\begin{description}
    \item[$\epsilon_{\text{trans}}$] Translational kinetic energy (J)
    \item[$m$] Mass of a molecule (kg)
    \item[$v_{\text{rms}}$] Root mean square velocity (m/s)
\end{description}

\subsection*{Average Translational Kinetic Energy}
\begin{equation}
\langle \epsilon_{\text{trans}} \rangle = \frac{3}{2} k_B T
\end{equation}
where:
\begin{description}
    \item[$\langle \epsilon_{\text{trans}} \rangle$] Average translational kinetic energy (J)
    \item[$k_B$] Boltzmann constant (J/K)
    \item[$T$] Temperature (K)
\end{description}

\subsection*{Thermal Energy of a Monatomic Gas}
\begin{equation}
E = \frac{3}{2} Nk_B T
\end{equation}
where:
\begin{description}
    \item[$E$] Thermal energy (J)
    \item[$N$] Number of molecules (mol)
    \item[$k_B$] Boltzmann constant (J/K)
    \item[$T$] Temperature (K)
\end{description}
end{equation}

\textbf{Number density:}
\begin{equation}
\frac{N}{V} = n
\end{equation}
where:
\begin{description}
    \item[$N$] Number of molecules (molecules)
    \item[$V$] Volume (m\(^3\))
    \item[$n$] Number density (molecules/m\(^3\))
\end{description}

\subsection*{Efficiency of a Carnot Engine}
\begin{equation}
n = 1 - \frac{T_c}{T_h}
\end{equation}
where:
\begin{description}
    \item[$n$] Efficiency
    \item[$T_c$] Temperature of the cold reservoir (K)
    \item[$T_h$] Temperature of the hot reservoir (K)
\end{description}
textbf{Note:} First change temperatures to Kelvin, then at the end convert back to Celsius. Also, \( n \) needs to be divided by 100 to remove percentage.

\subsection*{Specific Heat Capacity at Constant Volume for a Monatomic Gas}
\begin{equation}
C_V = \frac{3}{2}R = 12.5 \, \text{J/mol·K}
\end{equation}
where:
\begin{description}
    \item[$C_V$] Specific heat capacity at constant volume (J/mol·K)
    \item[$R$] Universal gas constant (J/mol·K)
\end{description}

\subsection*{Entropy of a Macrostate with Multiplicity}
\begin{equation}
S = k_B \ln \Omega
\end{equation}
where:
\begin{description}
    \item[$S$] Entropy (J/K)
    \item[$k_B$] Boltzmann constant (J/K)
    \item[$\Omega$] Multiplicity (number of microstates) (dimensionless)
\end{description}

\subsection*{Quasi-static Process}
\begin{equation}
dS = \frac{dQ}{T}
\end{equation}
where:
\begin{description}
    \item[$dS$] Change in entropy (J/K)
    \item[$dQ$] Heat added to the system (J)
    \item[$T$] Temperature (K)
\end{description}


\subsection*{Conclusion: Important Equations}
\begin{equation}
E_{\text{th, monatomic}} = \frac{3}{2} nRT
\end{equation}
\begin{equation}
E_{\text{th, diatomic}} = \frac{5}{2} nRT
\end{equation}
\begin{equation}
E_{\text{th, solid}} = 3nRT
\end{equation}
\begin{equation}
E_{\text{monatomic}} = \frac{3}{2} Nk_B T
\end{equation}
\begin{equation}
E_{\text{diatomic}} = \frac{5}{2} Nk_B T
\end{equation}
\begin{equation}
E_{\text{solid}} = 3Nk_B T
\end{equation}
\begin{equation}
C_{V, \text{monatomic}} = \frac{3}{2}R
\end{equation}
\textbf{Monatomic gas has \( C_V = \frac{3}{2}R \)}
\begin{equation}
C_{V, \text{diatomic}} = \frac{5}{2}R
\end{equation}
\textbf{Diatomic gas has \( C_V = \frac{5}{2}R \)}
\begin{equation}
C_{V, \text{solid}} = 3R
\end{equation}
\textbf{Elemental solid has \( C_V = 3R \)}

\end{document}