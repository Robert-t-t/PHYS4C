\documentclass{article}
\usepackage{amsmath, amssymb}

\begin{document}

\title{Thermodynamics Formulas - Chapters 19 and 20}
\author{}
\date{}
\maketitle

\section{First Law of Thermodynamics}
\textbf{Energy Conservation:}
\begin{equation}
\Delta E_{\text{th}} = W + Q
\end{equation}
where:
\begin{itemize}
    \item $\Delta E_{\text{th}}$ = Change in thermal energy (J)
    \item $W$ = Work done on the system (J)
    \item $Q$ = Heat added to the system (J)
\end{itemize}

\subsection{Work Done on an Ideal Gas}
\begin{equation}
W = - \int_{V_i}^{V_f} P \, dV
\end{equation}

\subsection{Calorimetry Equation (Heat Transfer)}
\begin{equation}
Q = mc \Delta T
\end{equation}
where:
\begin{itemize}
    \item $Q$ = Heat energy transferred (J)
    \item $m$ = Mass (kg)
    \item $c$ = Specific heat capacity (J/kg$\cdot$K)
    \item $\Delta T$ = Temperature change (K)
\end{itemize}

\subsection{Heat of Transformation (Phase Change)}
\begin{equation}
Q = \pm m L
\end{equation}
where:
\begin{itemize}
    \item $L$ = Latent heat (J/kg)
    \item $L_f$ = Heat of fusion (solid $\leftrightarrow$ liquid)
    \item $L_v$ = Heat of vaporization (liquid $\leftrightarrow$ gas)
\end{itemize}

\section{Heat Transfer Mechanisms}
\subsection{Conduction}
\begin{equation}
\frac{dQ}{dt} = k A \frac{\Delta T}{L}
\end{equation}

\subsection{Radiation}
\begin{equation}
\frac{dQ}{dt} = e \sigma A T^4
\end{equation}

\section{Ideal Gases and Heat}
\subsection{Ideal Gas Law}
\begin{equation}
PV = nRT
\end{equation}

\subsection{Internal Energy of an Ideal Gas}
\begin{equation}
E_{\text{th}} = \frac{f}{2} nRT
\end{equation}

\subsection{Root-Mean-Square (RMS) Speed}
\begin{equation}
v_{\text{rms}} = \sqrt{\frac{3 k_B T}{m}}
\end{equation}

\section{Thermodynamic Processes}
\textbf{Isothermal Process:}
\begin{equation}
W = - n R T \ln \left( \frac{V_f}{V_i} \right)
\end{equation}

\textbf{Adiabatic Process:}
\begin{equation}
PV^{\gamma} = \text{constant}
\end{equation}
\begin{equation}
TV^{\gamma - 1} = \text{constant}
\end{equation}

\textbf{Heat Capacities:}
\begin{equation}
C_P = C_V + R
\end{equation}

For a monatomic gas:
\begin{equation}
C_V = \frac{3}{2} R, \quad C_P = \frac{5}{2} R
\end{equation}

For a diatomic gas:
\begin{equation}
C_V = \frac{5}{2} R, \quad C_P = \frac{7}{2} R
\end{equation}


\end{document}

